\newcommand{\pulltf}[3]{

\begin{frame}{Pulls for #3}
\begin{figure}
\includegraphics[width=\textwidth]{2020_05_19/antiSym_#1/3/pulls/#2.png}
\caption{Blue : $\pm 1$, Green : $\pm \frac{1}{\sqrt{N}}$}
\end{figure}
\end{frame}
}
\newcommand{\pulldisttf}[3]{

\begin{frame}{Distributions for #3}

\begin{figure}
\includegraphics[width=\textwidth]{2020_05_19/antiSym_#1/3/pulls/pulls/#2_00.png}
\caption{$C_{00}$}
\end{figure}
\end{frame}

\begin{frame}{Distributions for #3}
\begin{figure}
\includegraphics[width=\textwidth]{2020_05_19/antiSym_#1/3/pulls/pulls/#2_01.png}
\caption{$C_{01}$}
\end{figure}
\end{frame}

\begin{frame}{Distributions for #3}
\begin{figure}
\includegraphics[width=\textwidth]{2020_05_19/antiSym_#1/3/pulls/pulls/#2_10.png}
\caption{$C_{10}$}
\end{figure}
\end{frame}

\begin{frame}{Distributions for #3}
\begin{figure}
\includegraphics[width=\textwidth]{2020_05_19/antiSym_#1/3/pulls/pulls/#2_11.png}
\caption{$C_{11}$}
\end{figure}
\end{frame}

\begin{frame}{Distributions for #3}
\begin{figure}
\includegraphics[width=\textwidth]{2020_05_19/antiSym_#1/3/pulls/pulls/#2_02.png}
\caption{$C_{02}$}
\end{figure}
\end{frame}

\begin{frame}{Distributions for #3}
\begin{figure}
\includegraphics[width=\textwidth]{2020_05_19/antiSym_#1/3/pulls/pulls/#2_20.png}
\caption{$C_{20}$}
\end{figure}
\end{frame}
}


\newcommand{\statustf}[3]{
\begin{frame}{Statuses for #3}
\begin{figure}
\includegraphics[width=\textwidth]{2020_05_19/antiSym_#1/3/#2_Status.png}
\end{figure}
\end{frame}

}


\newcommand{\pullstf}[2]{
\pulltf{#1}{Comb}{Combined Tags using #2}
\pulltf{#1}{KK}{$\KK$ Tag using #2}
\pulltf{#1}{Kspi0}{$\Kspiz$ Tag using #2}
\pulltf{#1}{Kppim}{$\Kppim$ Tag using #2}
\pulltf{#1}{Kmpip}{$\Kmpip$ Tag using #2}
\pulltf{#1}{Kspipi}{$\Kspipi$ Tag using #2}

}

\newcommand{\statstf}[2]{
\statustf{#1}{Comb}{Combined Tags using #2}
\statustf{#1}{KK}{$\KK$ Tag using #2}
\statustf{#1}{Kspi0}{$\Kspiz$ Tag using #2}
\statustf{#1}{Kppim}{$\Kppim$ Tag using #2}
\statustf{#1}{Kmpip}{$\Kmpip$ Tag using #2}
\statustf{#1}{Kspipi}{$\Kspipi$ Tag using #2}

}


\newcommand{\pulldiststf}[2]{
\pulldisttf{#1}{Comb}{Combined Tags using #2}
\pulldisttf{#1}{KK}{$\KK$ Tag using #2}
\pulldisttf{#1}{Kspi0}{$\Kspiz$ Tag using #2}
\pulldisttf{#1}{Kppim}{$\Kppim$ Tag using #2}
\pulldisttf{#1}{Kmpip}{$\Kmpip$ Tag using #2}
\pulldisttf{#1}{Kspipi}{$\Kspipi$ Tag using #2}

}
\begin{frame}{Overview}
    \begin{itemize}
	\item Performed a pull study for $C_{00..03,00..30}$
            \item \texttt{Singularity} image available at \texttt{/eos/lhcb/user/j/jolane/ampgen.sif} to run \texttt{AmpGen} programs.
            \item Made the object \texttt{pCoherentSum} which constructs the ampltidues for $B \to (\Dz,\Dzbar) h$ analyses.
            \item Built \texttt{gammaFit} program to perform simulatenous fit of \texttt{BESIII} data with \texttt{pCorrelatedSum} objects and \texttt{LHCb} data with \texttt{pCoherentSum} objects.
    \end{itemize}
\end{frame}

\pullstf{legendre}{Legendre antisymmetric}
\statstf{legendre}{Legendre antisymmetric}

\begin{frame}{\texttt{pCoherentSum}}
        For the decay $B^- \to D h^-$, we have the amplitude,
	\begin{equation}
                        A_B(\z) = A_{\Dz}(\z) + r_B e^{i \left(\delta_B - \gamma\right)} A_{\Dzbar}(\z), 
	\end{equation}
        \begin{equation}
                \begin{split}
			|A_B(\z)|^2 &= |A_{\Dz}(\z)|^2 + r_B^2 |A_{\Dzbar}(\z)|^2 \\ &+ 2 r_B |A_{\Dz}(\z)||A_{\Dzbar}(\z)| \cos \left(\dd(\z) + \delta_B - \gamma\right)
                \end{split}
        \end{equation}
        We build \texttt{pCoherentSum} with \texttt{CoherentSum}'s for $A_{\Dz, \Dzbar}$, and the 3 real parameters $r_B, \delta_B, \gamma$ are \texttt{MinuitParameter}'s. 
        The $\dd$ is corrected by the same $C_{ij}$ as the amplitude for correlated amplitudes.
        In the case of $B^+ \to \bar{D} h^+$, we change $\Dz$ to $\Dzbar$ and change the $-\gamma$ to $+\gamma$, the $r_B$ and $\delta_B$ are the same \texttt{MinuitParameter}'s.
        
   \end{frame}
   
\begin{frame}{Testing \texttt{pCoherentSum}}
	\begin{itemize}
		\item We see if $r_B e^{i (\delta_B - \gamma)} =0$ reproduces the single $\Dz \to \kspipi$ amplitude we get from \texttt{CoherentSum}
		\item We see if $r_B e^{i (\delta_B - \gamma)} =\pm1$ reproduces $D \to \kspipi$ against $D \to \KK$ (-1) and $D \to \Kspiz$ (+1)
		\item We also check if the combined log likelihood for different $B \to D h$ decays works properly
		\item Finally we perform a pull study for just $r_B, \delta_B, \gamma$ as variables
	\end{itemize}
\end{frame}

\newcommand{\comparetf}[2]{
	\begin{frame}{#2}
		\includegraphics[width=\textwidth]{2020_05_25/figs/gamma/#1.png}
	\end{frame}
}


\comparetf{uncorrVStest0}{\texttt{CoherentSum} v.s. $r_B e^{i(\delta_B + \gamma)} = 0$}
\comparetf{KKVStestKK}{$\KK$ tag v.s. $r_B e^{i(\delta_B + \gamma)} = -1$}
\comparetf{Kspi0VStestKspi0}{$\Kspiz$ tag v.s. $r_B e^{i(\delta_B + \gamma)} = 1$}
\comparetf{rB_Norm}{Scan for $r_B$}
\comparetf{dB_Norm}{Scan for $\delta_B$}
\comparetf{gamma_Norm}{Scan for $\gamma$}
\comparetf{pullTest}{Pulls $ r_B, \delta_B, \gamma$}

   \begin{frame}{Simulatenous fit}
           We have 2 sets of data, one from \texttt{BESIII}, and one from \texttt{LHCb}. We build a set of correlated amplitudes
           \begin{equation}
                   \Psi_\text{T}(\z_1, \z_2) = A_{\Dz}(\z_1) A_{\bar{T}}(\z_2) - A_{\Dzbar}(\z_1) A_{T}(\z_2),
           \end{equation}
            for tags, $T$. We can make a set of log likelihoods for each $T$ (with tagged $\Dz \to \kspipi$ v.s. $\Dzbar \to T$),
            \begin{equation}
                    \log \mathcal{L}_{T} = \sum_{\z_{1,2} \in D_T} \log |\Psi_\text{T}(\z_1,\z_2)|^2.
            \end{equation}
            Next for the all of our $B \to Dh$ decays (e.g. $B^\pm \to D h^\pm$) we build log-likelihoods,
            \begin{equation}
                    \log \mathcal{L}_{B} = \sum_{\z \in D_B} \log |A_B(\z)|^2,
            \end{equation}
            finally we construct the total log-likelihood, 
            \begin{equation}
                    \log \mathcal{L}_\text{total} = \sum_T \log \mathcal{L}_{T} + \sum_B \log \mathcal{L}_{B},
            \end{equation}

   \end{frame}



\begin{frame}{Yields}
From \url{https://arxiv.org/pdf/2003.00091.pdf}
\begin{tabular}{|c|c|c|}
Tag & Yield from \SI{2.9}{\per\femto\barn} at BESIII & $10 \times$ yield of BESIII \\ \hline
$\KK$ & 443 & 4430 \\
$\Kspiz$ & 643 & 6430 \\
$\Kppim$ & 4740 & 47400 \\
$\Kspipi$ & 899 & 8990
\end{tabular}
	From \url{https://ora.ox.ac.uk/objects/uuid:c9744e03-1e6d-4c1a-8724-b58d8005ca56}
	\begin{tabular}{|c|c|c|}
		B Decay & Yield from \SI{2}{\per\femto\barn} at LHCb & $5 \times $ yield \\ \hline 
		$B^- \to D h^-$ & 1917 & 9585 \\
		$B^+ \to D h^+$ & 1940 & 9700
	\end{tabular}
\end{frame}



