\begin{frame}{Overview}
    \begin{itemize}
        \item Likelihood scan performed for a variety of tags, $\KK, \Kppim, \Kmpip, \Kspiz, \Kspipi$ and the combined likelihood for the ``CP'' polynomial.
        \item The combined likelihood needed to be remade (backup slides + appendix in note). In short there was a bug in the C++ code which made the initial implementation of the combined likelihood crash, this new implementation does not crash. 
        \item The time taken for the scan was longer due to this problem + the length of time to perform the scans.
        \item Some parameters aren't included since the plots aren't shown because the scan crashed.
    \end{itemize}
\end{frame}

\begin{frame}{Likelihood Scan for $C_{00}^{-}$}
\begin{figure}
    \centering
        \includegraphics[width=\textwidth]{2020_04_23/figs/M00_Norm.png}
    \caption{Likelihood Scan for $C_{00}^{-}$, where the $y$ axis measures the `distance' of the log likelihood at one value of $C_{10}^{-}$ from the smallest log-likelihood in the sample.}
    \label{fig:scanCM00}
\end{figure}
\end{frame}


\begin{frame}{Likelihood Scan for $C_{01}^{-}$}
\begin{figure}
    \centering
        \includegraphics[width=\textwidth]{2020_04_23/figs/M01_Norm.png}
    \caption{Likelihood Scan for $C_{01}^{-}$, where the $y$ axis measures the `distance' of the log likelihood at one value of $C_{10}^{-}$ from the smallest log-likelihood in the sample.}
    \label{fig:scanCM01}
\end{figure}
\end{frame}

\begin{frame}{Likelihood Scan for $C_{10}^{-}$}
\begin{figure}
    \centering
        \includegraphics[width=\textwidth]{2020_04_23/figs/M10_Norm.png}
    \caption{Likelihood Scan for $C_{10}^{-}$, where the $y$ axis measures the `distance' of the log likelihood at one value of $C_{10}^{-}$ from the smallest log-likelihood in the sample.}
    \label{fig:scanCM10}
\end{figure}
\end{frame}

\begin{frame}{Likelihood Scan for $C_{02}^{-}$}
\begin{figure}
    \centering
        \includegraphics[width=\textwidth]{2020_04_23/figs/M02_Norm.png}
    \caption{Likelihood Scan for $C_{02}^{-}$, where the $y$ axis measures the `distance' of the log likelihood at one value of $C_{10}^{-}$ from the smallest log-likelihood in the sample.}
    \label{fig:scanCM02}
\end{figure}
\end{frame}

\begin{frame}{Likelihood Scan for $C_{11}^{-}$}
\begin{figure}
    \centering
        \includegraphics[width=\textwidth]{2020_04_23/figs/M11_Norm.png}
    \caption{Likelihood Scan for $C_{11}^{-}$, where the $y$ axis measures the `distance' of the log likelihood at one value of $C_{10}^{-}$ from the smallest log-likelihood in the sample.}
    \label{fig:scanCM11}
\end{figure}
\end{frame}


\begin{frame}{Likelihood Scan for $C_{20}^{-}$}
\begin{figure}
    \centering
        \includegraphics[width=\textwidth]{2020_04_23/figs/M20_Norm.png}
    \caption{Likelihood Scan for $C_{20}^{-}$, where the $y$ axis measures the `distance' of the log likelihood at one value of $C_{10}^{-}$ from the smallest log-likelihood in the sample.}
    \label{fig:scanCM20}
\end{figure}
\end{frame}





\begin{frame}{Likelihood Scan for $C_{00}^{+}$}
\begin{figure}
    \centering
        \includegraphics[width=\textwidth]{2020_04_23/figs/P00_Norm.png}
    \caption{Likelihood Scan for $C_{00}^{+}$, where the $y$ axis measures the `distance' of the log likelihood at one value of $C_{00}^{+}$ from the smallest log-likelihood in the sample.}
    \label{fig:scanCP01}
\end{figure}
\end{frame}

\begin{frame}{Likelihood Scan for $C_{01}^{+}$}
\begin{figure}
    \centering
        \includegraphics[width=\textwidth]{2020_04_23/figs/P01_Norm.png}
    \caption{Likelihood Scan for $C_{01}^{+}$, where the $y$ axis measures the `distance' of the log likelihood at one value of $C_{10}^{+}$ from the smallest log-likelihood in the sample.}
    \label{fig:scanCP01}
\end{figure}
\end{frame}

\begin{frame}{Likelihood Scan for $C_{10}^{+}$}
\begin{figure}
    \centering
        \includegraphics[width=\textwidth]{2020_04_23/figs/P10_Norm.png}
    \caption{Likelihood Scan for $C_{10}^{+}$, where the $y$ axis measures the `distance' of the log likelihood at one value of $C_{10}^{+}$ from the smallest log-likelihood in the sample.}
    \label{fig:scanCP10}
\end{figure}
\end{frame}

\begin{frame}{Likelihood Scan for $C_{02}^{+}$}
\begin{figure}
    \centering
        \includegraphics[width=\textwidth]{2020_04_23/figs/P02_Norm.png}
    \caption{Likelihood Scan for $C_{02}^{+}$, where the $y$ axis measures the `distance' of the log likelihood at one value of $C_{02}^{+}$ from the smallest log-likelihood in the sample.}
    \label{fig:scanCP02}
\end{figure}
\end{frame}

\begin{frame}{Likelihood Scan for $C_{11}^{+}$}
\begin{figure}
    \centering
        \includegraphics[width=\textwidth]{2020_04_23/figs/P11_Norm.png}
    \caption{Likelihood Scan for $C_{11}^{+}$, where the $y$ axis measures the `distance' of the log likelihood at one value of $C_{11}^{+}$ from the smallest log-likelihood in the sample.}
    \label{fig:scanCP11}
\end{figure}
\end{frame}


\begin{frame}{Likelihood Scan for $C_{20}^{+}$}
\begin{figure}
    \centering
        \includegraphics[width=\textwidth]{2020_04_23/figs/P20_Norm.png}
    \caption{Likelihood Scan for $C_{20}^{+}$, where the $y$ axis measures the `distance' of the log likelihood at one value of $C_{20}^{+}$ from the smallest log-likelihood in the sample.}
    \label{fig:scanCP20}
\end{figure}
\end{frame}

\begin{frame}
Backup Slides
\end{frame}

\begin{frame}{Likelihood Scan for $C_{00}^{-}$}
\begin{figure}
    \centering
        \includegraphics[width=\textwidth]{2020_04_23/figs/M00.png}
    \caption{Likelihood Scan for $C_{00}^{-}$, where the $y$ axis measures the `distance' of the log likelihood at one value of $C_{00}^{-}$.}
    \label{fig:scanCM01}
\end{figure}
\end{frame}

\begin{frame}{Likelihood Scan for $C_{01}^{-}$}
\begin{figure}
    \centering
        \includegraphics[width=\textwidth]{2020_04_23/figs/M01.png}
    \caption{Likelihood Scan for $C_{01}^{-}$, where the $y$ axis measures the `distance' of the log likelihood at one value of $C_{10}^{-}$.}
    \label{fig:scanCM01}
\end{figure}
\end{frame}

\begin{frame}{Likelihood Scan for $C_{10}^{-}$}
\begin{figure}
    \centering
        \includegraphics[width=\textwidth]{2020_04_23/figs/M10.png}
    \caption{Likelihood Scan for $C_{10}^{-}$, where the $y$ axis measures the `distance' of the log likelihood at one value of $C_{10}^{-}$.}
    \label{fig:scanCM10}
\end{figure}
\end{frame}

\begin{frame}{Likelihood Scan for $C_{02}^{-}$}
\begin{figure}
    \centering
        \includegraphics[width=\textwidth]{2020_04_23/figs/M02.png}
    \caption{Likelihood Scan for $C_{02}^{-}$, where the $y$ axis measures the `distance' of the log likelihood at one value of $C_{02}^{-}$.}
    \label{fig:scanCM02}
\end{figure}
\end{frame}

\begin{frame}{Likelihood Scan for $C_{11}^{-}$}
\begin{figure}
    \centering
        \includegraphics[width=\textwidth]{2020_04_23/figs/M11.png}
    \caption{Likelihood Scan for $C_{11}^{-}$, where the $y$ axis measures the `distance' of the log likelihood at one value of $C_{11}^{-}$.}
    \label{fig:scanCM11}
\end{figure}
\end{frame}


\begin{frame}{Likelihood Scan for $C_{20}^{-}$}
\begin{figure}
    \centering
        \includegraphics[width=\textwidth]{2020_04_23/figs/P20.png}
    \caption{Likelihood Scan for $C_{20}^{-}$, where the $y$ axis measures the `distance' of the log likelihood at one value of $C_{20}^{-}$.}
    \label{fig:scanCM20}
\end{figure}
\end{frame}



\begin{frame}{Likelihood Scan for $C_{00}^{+}$}
\begin{figure}
    \centering
        \includegraphics[width=\textwidth]{2020_04_23/figs/P00.png}
    \caption{Likelihood Scan for $C_{00}^{+}$, where the $y$ axis measures the `distance' of the log likelihood at one value of $C_{00}^{+}$.}
    \label{fig:scanCP01}
\end{figure}
\end{frame}

\begin{frame}{Likelihood Scan for $C_{01}^{+}$}
\begin{figure}
    \centering
        \includegraphics[width=\textwidth]{2020_04_23/figs/P01.png}
    \caption{Likelihood Scan for $C_{01}^{+}$, where the $y$ axis measures the `distance' of the log likelihood at one value of $C_{10}^{+}$.}
    \label{fig:scanCP01}
\end{figure}
\end{frame}

\begin{frame}{Likelihood Scan for $C_{10}^{+}$}
\begin{figure}
    \centering
        \includegraphics[width=\textwidth]{2020_04_23/figs/P10.png}
    \caption{Likelihood Scan for $C_{10}^{+}$, where the $y$ axis measures the `distance' of the log likelihood at one value of $C_{10}^{+}$.}
    \label{fig:scanCP10}
\end{figure}
\end{frame}

\begin{frame}{Likelihood Scan for $C_{02}^{+}$}
\begin{figure}
    \centering
        \includegraphics[width=\textwidth]{2020_04_23/figs/P02.png}
    \caption{Likelihood Scan for $C_{02}^{+}$, where the $y$ axis measures the `distance' of the log likelihood at one value of $C_{02}^{+}$.}
    \label{fig:scanCP02}
\end{figure}
\end{frame}

\begin{frame}{Likelihood Scan for $C_{11}^{+}$}
\begin{figure}
    \centering
        \includegraphics[width=\textwidth]{2020_04_23/figs/P11.png}
    \caption{Likelihood Scan for $C_{11}^{+}$, where the $y$ axis measures the `distance' of the log likelihood at one value of $C_{11}^{+}$.}
    \label{fig:scanCP11}
\end{figure}
\end{frame}


\begin{frame}{Likelihood Scan for $C_{20}^{+}$}
\begin{figure}
    \centering
        \includegraphics[width=\textwidth]{2020_04_23/figs/P20.png}
    \caption{Likelihood Scan for $C_{20}^{+}$, where the $y$ axis measures the `distance' of the log likelihood at one value of $C_{20}^{+}$.}
    \label{fig:scanCP20}
\end{figure}
\end{frame}



\begin{frame}{Time to perform Likelihood Scan}
we need to calculate 
\begin{equation}
   N_\text{int} = 2 \times N_\text{tag} \times N_\text{param}  = 2 \times N_\text{tag} \times (N_\text{Order} + 1) (N_\text{Order} + 2)
\end{equation}
For $N_\text{Order} = 2$ this is $2 \times 5 \times 3 \times 4 = 120$. Each integral takes $\sim \mathcal{O}(1)m$, i.e. for all of our tags we would expect up to $\sim 30m - 2h$ per value - i.e. it would take anywhere from $5h - 20h$ to make a full plot for every parameter. If we run each job on a $\mathcal{O}(10)$ core computer, we can hope to reduce this time by a factor $10$, i.e. instead of $\mathcal{O}(10)h$ we could do it in $\mathcal{O}(1)$h.
When testing we can reduce this time by a few factors - e.g. only using 2-3 tags and with only 2-3 different values for one parameter would only take $2 \times 3 \times 1 \times \mathcal{O}(1) \sim \mathcal{O}(5)$ minutes. 
\end{frame}
\begin{frame}{From Appendix of the note}
We wish to minimise the function
\begin{equation}
    -2\log\mathcal{L}_\text{comb} \equiv \sum_\text{tag} \left(-2 \log \mathcal{L}_\text{tag}\right) \label{eqn:combLL},
\end{equation}
where we construct a log-likelihood for each tag.

To perform this sum in \texttt{AmpGen} we have two options
\begin{enumerate}
    \item Make a \texttt{std::vector} of $-2\log\mathcal{L}_\text{tag}$ objects, \texttt{CorrelatedLL}, \texttt{\{CorrLL\_KK, CorrLL\_Kppim,...\}}, and calculate (\ref{eqn:combLL}),
    \item Make the object that calculates (\ref{eqn:combLL}) with \texttt{std::vector} versions of the input to \texttt{CorrelatedLL},
\end{enumerate}
in the first case, we found that \texttt{AmpGen} crashes when initialising the combined log-likelihood - this is likely an issue with memory management in the \texttt{C++} code implementing this combined object. 
\end{frame}
\begin{frame}


The second solution should avoid this by using only one instance of \texttt{MPS}, we build an object with the following inputs
\begin{enumerate}
    \item Four \texttt{std::vector}'s of \texttt{EventList}'s, \texttt{\{ SigData\_KK, SigData\_Kppim, ... \}} and \texttt{\{ TagData\_KK, TagData\_Kppim, ... \}} for the ``Data'' events (that we wish to fit) for both ``Signal'' (\texttt{Sig}) and ``tag'' (\texttt{Tag}) and \texttt{\{ SigInt\_KK, SigInt\_Kppim, ... \}} and \texttt{\{ TagInt\_KK, TagInt\_Kppim, ... \}} for the ``Integration'' events - which (for the moment) is needed as an input to normalise the correlated amplitude for each tag.
    \item Two \texttt{std::vector}'s of \texttt{EventType}'s, \texttt{\{SigType\_KK, SigType\_Kppim, ...\}} and \texttt{\{TagType\_KK, TagType\_Kppim, ...\}},
    \item A single \texttt{MinuitParameterSet} object that contains all of the information for every amplitude we intend to build.
\end{enumerate}
\end{frame}
\begin{frame}
We then build the combined log-likelihood in a single object by evaluating the sum
\begin{equation}
    \begin{split}
    -2\log\mathcal{L}_\text{comb} &=-2 \sum_\text{tag} \sum_{\z_1,\z_2 \in \textbf{Z}^\text{tag}_1, \textbf{Z}^\text{tag}_2} \log\left|\Psi_\text{tag}(\z_1, \z_2)\right|^2, \\ 
    \mathbf{Z}^\text{tag}_1, \mathbf{Z}^\text{tag}_2 &= \texttt{SigData\_tag}, \texttt{TagData\_tag},\label{eqn:fullCombLL}
    \end{split}
\end{equation}
now any change to \texttt{MPS} will cause \texttt{AmpGen} to recalculate the sum (\ref{eqn:fullCombLL}) , and any changes to the $\Psi_\text{tag}$ are also reevaluated (including changes to the normalisation of $\Psi_\text{tag}$).
\end{frame}
